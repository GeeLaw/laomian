\chapter{Some Chapter}

\section{Some Section}

The inter-word and inter-sentence spacings have been fixed on 3 July 2024.
If you upgrade from an earlier version,
the typesetting result might change dramatically.
Use \texttt{\string\willstop} and \texttt{\string\wontstop}
before a punctuation mark
to avoid kerning issues with \texttt{\string\@}.

In the following lines, the examples are
no control sequence,
\texttt{\string\willstop},
\texttt{\string\wontstop},
and repeat.

\frenchspacing
(French spacing)\\
{\ttfamily
Code:\\
Hello, WORLD world. Hello, world!\\
Hello, WORLD world\willstop. Hello, world!\\
Hello, WORLD world\wontstop. Hello, world!\\
{\sffamily
Sans Serif:\\
Hello, WORLD world. Hello, world!\\
Hello, WORLD world\willstop. Hello, world!\\
Hello, WORLD world\wontstop. Hello, world!\\
Hello, world WORLD. Hello, world!\\
Hello, world WORLD\willstop. Hello, world!\\
Hello, world WORLD\wontstop. Hello, world!\\
}%
Code:\\
Hello, world WORLD. Hello, world!\\
Hello, world WORLD\willstop. Hello, world!\\
Hello, world WORLD\wontstop. Hello, world!\\
}%
Serif:\\
Hello, WORLD world. Hello, world!\\
Hello, WORLD world\willstop. Hello, world!\\
Hello, WORLD world\wontstop. Hello, world!\\
Hello, world WORLD. Hello, world!\\
Hello, world WORLD\willstop. Hello, world!\\
Hello, world WORLD\wontstop. Hello, world!

\nonfrenchspacing
(non-French spacing, the default)\\
{\ttfamily
Code:\\
Hello, WORLD world. Hello, world!\\
Hello, WORLD world\willstop. Hello, world!\\
Hello, WORLD world\wontstop. Hello, world!\\
{\sffamily
Sans Serif:\\
Hello, WORLD world. Hello, world!\\
Hello, WORLD world\willstop. Hello, world!\\
Hello, WORLD world\wontstop. Hello, world!\\
Hello, world WORLD. Hello, world!\\
Hello, world WORLD\willstop. Hello, world!\\
Hello, world WORLD\wontstop. Hello, world!\\
}%
Code:\\
Hello, world WORLD. Hello, world!\\
Hello, world WORLD\willstop. Hello, world!\\
Hello, world WORLD\wontstop. Hello, world!\\
}%
Serif:\\
Hello, WORLD world. Hello, world!\\
Hello, WORLD world\willstop. Hello, world!\\
Hello, WORLD world\wontstop. Hello, world!\\
Hello, world WORLD. Hello, world!\\
Hello, world WORLD\willstop. Hello, world!\\
Hello, world WORLD\wontstop. Hello, world!

\subsection{Some Subsection}

This template is used to prepare manuscripts for works in cryptography.%
\footnote{This is a footnote.}

\subsubsection{Subsubsection.}
This is a run-in heading in bold face.
Normal,
\textit{Italics,}
\textsl{Slanted,}
\textbf{Bold,}
\textsc{Small-Caps,}
\textit{\textbf{Italics Bold,}}
\textsl{\textbf{Slanted Bold,}}
\textsf{Sans Serif,}
\texttt{Monospaced.}

Avoid using \texttt{\string\begin\stringall{equation}\endstringall},
\texttt{\string\begin\stringall{gather}\endstringall},
\texttt{\string\begin\stringall{split}\endstringall},
and \texttt{\string$\string$}.
Always prefer \texttt{\string\begin\stringall{align}\endstringall}.
(One of) Euler's formula says
\begin{align*}
e^{i\pi}+1=0.
\end{align*}

\paragraph{Paragraph.}
This is an italicized run-in heading, and
\texttt{\string\subparagraph} is \emph{explicitly} undefined.
Springer requires that table captions be above the table
and figure captions be below the figure,
as exemplified in Table~\ref{tab:table1} and \Figure~\ref{fig:figure1}.

\begin{table}[htb]
% Fine-tune its position.
\vskip 2em\relax
\def\baselinestretch{1}\selectfont
\centering
\capstart
\caption{Formats.}
\label{tab:table1}
\begin{tabular}{ll}
\toprule
\texttt{format=...} & \textbf{purpose} \\
\midrule
\texttt{eprint-draft} & full version, draft \\
\texttt{eprint} &
full version (\href{https://eprint.iacr.org/}{eprint.iacr.org}) \\
\texttt{focs-submission-draft} &
submission (FOCS), draft \\
\texttt{focs-submission} &
submission (FOCS) \\
\texttt{lncs-submission-draft} &
submission (IACR conferences), draft \\
\texttt{lncs-submission} &
submission (IACR conferences) \\
\texttt{lncs-camera-ready-draft} &
camera-ready, draft \\
\texttt{lncs-camera-ready-iacr} &
PDF archived by the IACR \\
\texttt{lncs-camera-ready-reference} &
PDF for Springer's reference \\
\texttt{lncs-camera-ready} &
source code to be processed by Springer \\
\bottomrule
\end{tabular}
\end{table}

\begin{figure}
\def\baselinestretch{1}\selectfont
\centering
\capstart
\begin{tabular}{ll}
\toprule
\textbf{option} & \textbf{description} \\
\midrule
\texttt{layout=6x9} & use the 2022 layout \\
\texttt{layout=38x48} & use the 2023 layout \\
\texttt{layout=6.5x9} & use the FOCS submission layout \\
\texttt{pdf-metadata=yes} & write PDF metadata if non-anonymous \\
\texttt{pdf-metadata=no} & do not write PDF metadata \\
\texttt{opens-on=right} & make the main body start \\
& \qquad on an odd-numbered page \\
\texttt{opens-on=any} & do not make the main body start \\
& \qquad on an odd-numbered page \\
\texttt{bst=alpha} & use \texttt{alpha.bst} \\
\texttt{bst=splncs04} & use \texttt{splncs04.bst} \\
\texttt{vec=itbf} & ${\texttt{\stringall\vec{a}\endstringall}=\boldsymbol{a}}$ \\
\texttt{vec=bf} & ${\texttt{\stringall\vec{a}\endstringall}=\mathbf{a}}$ \\
\texttt{absurdum=hitwall} & ${\texttt{\string\absurdum}=\absurdumhitwall}$ \\
\texttt{absurdum=lightning} & ${\texttt{\string\absurdum}=\absurdumlightning}$ \\
\texttt{links=colorful} & use colorful links \\
\texttt{links=blue} & use blue links \\
\texttt{fonts=computer-modern} & use Computer Modern fonts \\
\texttt{fonts=source} &
use \href{https://github.com/adobe-fonts}{Adobe Source} fonts \\
\texttt{fonts=times} & use Times fonts \\
\texttt{fonts=palatino} & use Palatino fonts \\
\texttt{fonts=libertine} & use Libertine fonts \\
\texttt{fonts=utopia} & use Utopia fonts \\
\texttt{underlines=smart} & make underlines \smartunderline{skip} descenders \\
\texttt{underlines=dumb} &
make underlines \dumbunderline{stay} below descenders \\
\texttt{non-anonymous} & make the document non-anonymous \\
\texttt{no-page-limits} & mark the document \\
& \qquad as having no page limits \\
& \quad and do not mark appendix \\
& \qquad as ``supplementary materials'' \\
\texttt{no-table-of-contents} & do not show table of contents \\
\texttt{no-separate-title-page} & do not use a separate title page \\
\texttt{no-total-number-of-pages} & do not show the total number of pages \\
\texttt{no-lncs-array-table-margins} & do not use array/table inter-column \\
& \qquad margins from LNCS \\
\texttt{no-swapped-table-caption-margins} &
do not swap top/bottom margins \\
& \qquad for table captions \\
\bottomrule
\end{tabular}
\caption{Options for customizing the formats.
Not all options are effective for all formats, and
not all option combinations are possible.}
\label{fig:figure1}
\end{figure}

\chapter{Utilities}

The draft mode displays page borders,
which can be used to ensure that no content bleeds off the type center.

\section{Call-Out Boxes}

Use
\texttt{\stringall\defcallout\luoji{Ji}\endstringall}
\emph{in the preamble}
to make
\luoji{This is a note from Ji.
\luoji{Nested call-outs are converted to paragraphs inside the same box.}}
control sequence \texttt{\string\luoji} typeset a call-out box for Ji,
which is only shown in draft mode.
To make the control sequence
\texttt{\string\long} (accept \expandafter\texttt\expandafter{\string\par}),
use
\texttt{\stringall\long\defcallout\luoji{Ji}\endstringall}.
Using \texttt{\string\long} is recommended.

\luoji{Note can go on empty paragraph,
but it looks weird and will disturb vertical typesetting.}

It is strongly discouraged to put a call-out box on an empty paragraph
as it makes vertical typesetting dependent on whether it is draft or not.
Note that even if there are no call-out boxes,
the typesetting result could be different for draft and non-draft
due to arithmetic errors
(in the draft version,
the type center is moved, and
rounding results could change).

\luoji{Note can start a paragraph.}
The colors of the boxes rotate so that it is easy to track which is where.

\section
[\texorpdfstring{% TeX
Format- and Option-Dependent Control \underlined{Sequences}
}{% PDF string
Format- and Option-Dependent Control Sequences}]
{Format- and Option-Dependent Control \underlined{Sequences}}

\subsubsection{Underlines.}
By default, ePrint uses descender-\underlined{skipping} underlines.
To \smartunderline{always} skip the descenders,
use \texttt{\string\smartunderline}, and
to always \dumbunderline{stay} below the descenders,
use \texttt{\string\dumbunderline}.
The control sequences are \emph{robust} so that
they can be used in moving arguments.

\subsubsection{Names of Referenced Objects.}
\texttt{\string\Figure},
\texttt{\string\Equation},
\texttt{\string\Section},
\texttt{\string\Figures},
\texttt{\string\Equations}, and
\texttt{\string\Sections}
provide the correct form of references:
\begin{itemize}
\item
In LNCS, they are Fig., Eq., Sect., Figs., Eqs., Sects., respectively.
This is required by Springer.
\item
In ePrint, they are Figure, Equation, Section, Figures, Equations, Sections.
\item
In the current format, they are
``\Figure'', ``\Equation'', ``\Section'',
``\Figures'', ``\Equations'', ``\Sections''.
\end{itemize}
However, if these words appear at the beginning of a sentence,
LNCS requires that you use the full word of instead of the abbreviated form
(so at the beginning of a sentence,
just use the word, instead of the control sequence,
for Springer compliance).
Use
\texttt{\string\WideNarrow},
\texttt{\string\PageLimitsNoLimits},
\texttt{\string\BibAlphaNumeric},
to choose one from the two candidates depending on the current style.
The type center is \WideNarrow{wide}{narrow},
there \PageLimitsNoLimits{is a}{are no} page limit\PageLimitsNoLimits{}{s},
and the bibliography is \BibAlphaNumeric{alphanumeric}{numeric}.
These control sequences are \texttt{\string\long},
so they can accept multi-paragraph arguments.

\section{Mathematical Symbols}

All Greek letters
(including
\texttt{\string\omicron} and \texttt{\string\Omicron})
have their lowercase and uppercase control sequences,
and the uppercase letters are upright.
\begin{align*}
\Alpha,\alpha,{}&\vec{\Alpha},\vec{\alpha},&
\Beta,\beta,{}&\vec{\Beta},\vec{\beta},&
\Gamma,\gamma,{}&\vec{\Gamma},\vec{\gamma},&
\Delta,\delta,{}&\vec{\Delta},\vec{\delta},&
\Epsilon,\epsilon,{}&\vec{\Epsilon},\vec{\epsilon},&
\Zeta,\zeta,{}&\vec{\Zeta},\vec{\zeta},\\
\Eta,\eta,{}&\vec{\Eta},\vec{\eta},&
\Theta,\theta,{}&\vec{\Theta},\vec{\theta},&
\Iota,\iota,{}&\vec{\Iota},\vec{\iota},&
\Kappa,\kappa,{}&\vec{\Kappa},\vec{\kappa},&
\Lambda,\lambda,{}&\vec{\Lambda},\vec{\lambda},&
\Mu,\mu,{}&\vec{\Mu},\vec{\mu},
\displaybreak[3]\\
\Nu,\nu,{}&\vec{\Nu},\vec{\nu},&
\Xi,\xi,{}&\vec{\Xi},\vec{\xi},&
\Omicron,\omicron,{}&\vec{\Omicron},\vec{\omicron},&
\Pi,\pi,{}&\vec{\Pi},\vec{\pi},&
\Rho,\rho,{}&\vec{\Rho},\vec{\rho},&
\Sigma,\sigma,{}&\vec{\Sigma},\vec{\sigma},\\
\Tau,\tau,{}&\vec{\Tau},\vec{\tau},&
\Upsilon,\upsilon,{}&\vec{\Upsilon},\vec{\upsilon},&
\Phi,\phi,{}&\vec{\Phi},\vec{\phi},&
\Chi,\chi,{}&\vec{\Chi},\vec{\chi},&
\Psi,\psi,{}&\vec{\Psi},\vec{\psi},&
\Omega,\omega,{}&\vec{\Omega},\vec{\omega}.
\end{align*}
Control sequences similar to Office Equation Editor are provided:
\texttt{\string\doubleA}
for blackboard bold,
\texttt{\string\scriptA}
for usual handwritten,
\texttt{\string\cursiveA}
for cursive handwritten,
\texttt{\string\frakturA} and \texttt{\string\fraktura}
for fraktur.
All these letter control sequences
can be used as superscripts and subscripts
without being wrapped inside braces, i.e.,
it is correct to write
\texttt{\stringall$1^\cursiveA_\lambda$\endstringall}
for $1^\cursiveA_\lambda$.
\begin{align*}
\doubleA,\scriptA,\cursiveA,{}&\frakturA,\fraktura,&
\doubleB,\scriptB,\cursiveB,{}&\frakturB,\frakturb,&
\doubleC,\scriptC,\cursiveC,{}&\frakturC,\frakturc,&
\doubleD,\scriptD,\cursiveD,{}&\frakturD,\frakturd,\\
\doubleE,\scriptE,\cursiveE,{}&\frakturE,\frakture,&
\doubleF,\scriptF,\cursiveF,{}&\frakturF,\frakturf,&
\doubleG,\scriptG,\cursiveG,{}&\frakturG,\frakturg,
\displaybreak[3]\\
\doubleH,\scriptH,\cursiveH,{}&\frakturH,\frakturh,&
\doubleI,\scriptI,\cursiveI,{}&\frakturI,\frakturi,&
\doubleJ,\scriptJ,\cursiveJ,{}&\frakturJ,\frakturj,&
\doubleK,\scriptK,\cursiveK,{}&\frakturK,\frakturk,\\
\doubleL,\scriptL,\cursiveL,{}&\frakturL,\frakturl,&
\doubleM,\scriptM,\cursiveM,{}&\frakturM,\frakturm,&
\doubleN,\scriptN,\cursiveN,{}&\frakturN,\frakturn,
\displaybreak[3]\\
\doubleO,\scriptO,\cursiveO,{}&\frakturO,\frakturo,&
\doubleP,\scriptP,\cursiveP,{}&\frakturP,\frakturp,&
\doubleQ,\scriptQ,\cursiveQ,{}&\frakturQ,\frakturq,\\ &&
\doubleR,\scriptR,\cursiveR,{}&\frakturR,\frakturr,&
\doubleS,\scriptS,\cursiveS,{}&\frakturS,\frakturs,&
\doubleT,\scriptT,\cursiveT,{}&\frakturT,\frakturt,
\displaybreak[3]\\
\doubleU,\scriptU,\cursiveU,{}&\frakturU,\frakturu,&
\doubleV,\scriptV,\cursiveV,{}&\frakturV,\frakturv,&
\doubleW,\scriptW,\cursiveW,{}&\frakturW,\frakturw,\\ &&
\doubleX,\scriptX,\cursiveX,{}&\frakturX,\frakturx,&
\doubleY,\scriptY,\cursiveY,{}&\frakturY,\fraktury,&
\doubleZ,\scriptZ,\cursiveZ,{}&\frakturZ,\frakturz.
\end{align*}
\texttt{\string\1} creates a blackboard bold $\1$,
which can be used as the indicator symbol.
\texttt{\string\vec}
produces italicized boldfaced letters
(if the letters themselves are boldfaced
without \texttt{\string\vec}, of course),
which is the orthodox.
The orthodox can be switched off using option \texttt{vec=bf}.
It works with Greek letters
(lowercase Greek letters are never upright except for omicron),
as shown above.
Below is a table of \texttt{\string\vec}'s applied to Latin alphabet:
\begin{align*}
A,a,{}&\vec{A},\vec{a},&
B,b,{}&\vec{B},\vec{b},&
C,c,{}&\vec{C},\vec{c},&
D,d,{}&\vec{D},\vec{d},\\
E,e,{}&\vec{E},\vec{e},&
F,f,{}&\vec{F},\vec{f},&
G,g,{}&\vec{G},\vec{g},
\displaybreak[3]\\
H,h,{}&\vec{H},\vec{h},&
I,i,{}&\vec{I},\vec{i},&
J,j,{}&\vec{J},\vec{j},&
K,k,{}&\vec{K},\vec{k},\\
L,l,\ell,{}&\vec{L},\vec{l},\vec{\ell},&
M,m,{}&\vec{M},\vec{m},&
N,n,{}&\vec{N},\vec{n},
\displaybreak[3]\\
O,o,{}&\vec{O},\vec{o},&
P,p,{}&\vec{P},\vec{p},&
Q,q,{}&\vec{Q},\vec{q},\\ &&
R,r,{}&\vec{R},\vec{r},&
S,s,{}&\vec{S},\vec{s},&
T,t,{}&\vec{T},\vec{t},
\displaybreak[3]\\
U,u,{}&\vec{U},\vec{u},&
V,v,{}&\vec{V},\vec{v},&
W,w,{}&\vec{W},\vec{w},\\ &&
X,x,{}&\vec{X},\vec{x},&
Y,y,{}&\vec{Y},\vec{y},&
Z,z,{}&\vec{Z},\vec{z}.
\end{align*}
Several symbols are redefined to the sane versions:
\begin{align*}
\texttt{\string\epsilon}&{}=\epsilon,&
\texttt{\string\varepsilon}&{}=\varepsilon,&
\texttt{\string\dumbepsilon}&{}=\dumbepsilon,\\
\texttt{\string\phi}&{}=\phi,&
\texttt{\string\varphi}&{}=\varphi,&
\texttt{\string\dumbphi}&{}=\dumbphi,\\
\texttt{\string\emptyset}&{}=\emptyset,&
\texttt{\string\varnothing}&{}=\varnothing,&
\texttt{\string\dumbemptyset}&{}=\dumbemptyset.
\end{align*}
It is wise to add more tests here ($v\nu\upsilon u\mu$,
${p\concat q=1^{x\concat y^{z\concat w}}}$):
\begin{align*}
&n=\smallomega(\log n),
\WideNarrow{\qquad}{\quad}
n=\bigOmega(n^{1/2}),
\WideNarrow{\qquad}{\quad}
n=\smallo(2^n),
\WideNarrow{\qquad}{\quad}
n=\bigO(n^{3/2}),
\WideNarrow{\qquad}{\quad}
n=\bigTheta(n),\\
&\sum_{i=1}^n{i}=\frac{n(n+1)}{2},
\WideNarrow{\qquad}{\quad}
\lim_{n\to\infty}{\int_{n^{-1}}^1{x\,\mathrm{d}x}}=\frac12,
\WideNarrow{\qquad}{\quad}
\bigcap_{n\geq 1}{\biggl\{
x
\,\bigg|\,
x\in\bigcup_{m\geq n}{A_m}
\biggr\}}
=\limsup_{n\to\infty}{A_n},\\
&
\Pr\left[\begin{aligned}
k&{}\draws\mathsf{Gen}(1^\lambda)\\
c&{}\draws\mathsf{Enc}(1^\lambda,k,m)
\end{aligned}
\::\:
\mathsf{Dec}(1^\lambda,k,c)=m
\right]
=1-\negl(\lambda)
\quad\forall m\in\bit^{\poly(\lambda)}.
\end{align*}

\section{Theorem Environments}

Use
\texttt{\string\begin\stringall{restated}{Theorem~\ref{thm:blah}}\endstringall}
to restate a theorem or typeset an unnumbered theorem.

\begin{restated}{Theorem~\ref{thm:theorem10}}
This is a restatement.
\end{restated}

\begin{restated}
{Theorem~\ref{thm:theorem10}}
[name]
This is a restatement with name.
\end{restated}

\noindent
\texttt{\string\begin\stringall{redefined}{Definition~\ref{def:blah}}\endstringall}
to restate a definition or typeset an unnumbered definition.

\begin{redefined}{Some Definition}
This is a redefinition.
\end{redefined}

\begin{redefined}
{Some Definition}
[name]
This is a redefinition with name.
\end{redefined}

\noindent
Proofs and remarks are unnumbered.

\begin{proof}
This is \texttt{\string\begin\stringall{proof}\endstringall}.
\end{proof}

\begin{proof}
[Lemma~\ref{lem:lemma5}]
\HyperTargetToThisLine{pf:lemma5}%
This is
\begin{align*}
&\texttt{\string\begin\stringall{proof}\endstringall}\\%
&\texttt{[Lemma\stringall~\ref{lem:lemma5}\endstringall]}\\
&\texttt{\stringall\HyperTargetToThisLine{pf:lemma5}\endstringall\%}\\
&\texttt{...}\\
&\texttt{\string\end\stringall{proof}\endstringall}
\end{align*}
Use \texttt{\string\qedhere} to place the tombstone symbol
on the last line of a displayed equation,
for which an alignment tab is required.
\begin{align*}
a+b&{}=c.
\qedhere
\end{align*}
\end{proof}

\begin{proofsketch}
This is \texttt{\string\begin\stringall{proofsketch}\endstringall}.
\end{proofsketch}

\begin{proofsketch}
[Proposition~\ref{prop:proposition8}]
\HyperTargetToThisLine{pf:proposition8}%
The code of this proof sketch:
\begin{align*}
&\texttt{\string\begin\stringall{proofsketch}\endstringall}\\%
&\texttt{[Proposition\stringall~\ref{prop:proposition8}\endstringall]}\\
&\texttt{\stringall\HyperTargetToThisLine{pf:proposition8}\endstringall\%}\\
&\texttt{...}\\
&\texttt{\string\end\stringall{proofsketch}\endstringall}
\qedhere
\end{align*}
\end{proofsketch}

\noindent
The two \textit{reductio ad absurdum} symbols (hit-wall, lightning)
can always be accessed.
By default, \texttt{\string\absurdum} is \texttt{\string\absurdumhitwall}.
See the \texttt{absurdum=...} option.

\begin{proof}
In this proof,
\texttt{\stringall\let\qedsymbol\absurdum\endstringall}
[\absurdum, $\absurdum$]
(\absurdum) $(\absurdum)$.
\let\qedsymbol\absurdum
\end{proof}

\begin{proof}
In this proof,
\texttt{\stringall\let\qedsymbol\absurdumhitwall\endstringall}
[\absurdumhitwall, $\absurdumhitwall$]
(\absurdumhitwall) $(\absurdumhitwall)$.
\let\qedsymbol\absurdumhitwall
\end{proof}

\begin{proof}
In this proof,
\texttt{\stringall\let\qedsymbol\absurdumlightning\endstringall}
[\absurdumlightning, $\absurdumlightning$]
(\absurdumlightning) $(\absurdumlightning)$.
\let\qedsymbol\absurdumlightning
\end{proof}

\begin{remarks}
This is \texttt{\string\begin\stringall{remarks}\endstringall}.
\end{remarks}

\begin{remarks}
[name]
This is \texttt{\string\begin\stringall{remarks}[name]\endstringall}.
\end{remarks}

\noindent
However, \texttt{\string\begin\stringall{remark}\endstringall} (singular form) is numbered.

\begin{remark}
This is \texttt{\string\begin\stringall{remark}\endstringall}.
\end{remark}

\begin{remark}
[name]
This is \texttt{\string\begin\stringall{remark}[name]\endstringall}.
\end{remark}

\noindent
No theorem environment uses a nested counter.

\begin{definition}
This is \texttt{\string\begin\stringall{definition}\endstringall}.
\end{definition}

\begin{definition}
[name]
This is \texttt{\string\begin\stringall{definition}[name]\endstringall}.
In LNCS formats, definitions are italicized,
making it tiring to read a security experiment.
Use
\texttt{\string\begin\stringall{itemize}\upshape\item\textbf{Phase.}\endstringall}
to write them:
\begin{itemize}\upshape
\item\textbf{Setup.} Blah.
\item\textbf{Challenge.} Bruh.
\item\textbf{Guess.} Bleh.
\end{itemize}
\end{definition}

\begin{question}
This is \texttt{\string\begin\stringall{question}\endstringall}.
\end{question}

\begin{question}
[name]
This is \texttt{\string\begin\stringall{question}[name]\endstringall}.
\end{question}

\noindent
The style of definitions and questions depends on
whether the layout is LNCS or ePrint.

\begin{conjecture}
This is \texttt{\string\begin\stringall{conjecture}\endstringall}.
\end{conjecture}

\begin{conjecture}
[name]
This is \texttt{\string\begin\stringall{conjecture}[name]\endstringall}.
\end{conjecture}

\noindent
Corollaries, claims, lemmata, propositions, and theorems
share the same counter.

\begin{corollary}
This is \texttt{\string\begin\stringall{corollary}\endstringall}.
\end{corollary}

\begin{corollary}
[name]
This is \texttt{\string\begin\stringall{corollary}[name]\endstringall}.
\end{corollary}

\begin{claim}
This is \texttt{\string\begin\stringall{claim}\endstringall}.
\end{claim}

\begin{claim}
[name]
This is \texttt{\string\begin\stringall{claim}[name]\endstringall}.
\end{claim}

\begin{lemma}
[\hyperlink{pf:lemma5}{\P}]
\label{lem:lemma5}
This is
\texttt{\string\begin\stringall{lemma}[\hyperlink{pf:lemma5}{\P}]\endstringall}.
It is a good idea to create two-way link between
a proposition and its proof.
\end{lemma}

\begin{lemma}
This is \texttt{\string\begin\stringall{lemma}\endstringall}.
\end{lemma}

\begin{proposition}
This is \texttt{\string\begin\stringall{proposition}\endstringall}.
\end{proposition}

\begin{proposition}
[\hyperlink{pf:proposition8}{\P}]
\label{prop:proposition8}
The code of this proposition:
\begin{align*}
&\texttt{\string\begin\stringall{proposition}\endstringall}\\
&\texttt{[\stringall\hyperlink{pf:proposition8}{\P}\endstringall]}\\
&\texttt{...}\\
&\texttt{\string\end\stringall{proposition}\endstringall}
\end{align*}
\end{proposition}

\begin{theorem}
This is \texttt{\string\begin\stringall{theorem}\endstringall}.
\end{theorem}

\begin{theorem}
[name]
\label{thm:theorem10}
This is \texttt{\string\begin\stringall{theorem}[name]\endstringall}.
\end{theorem}

\noindent
Assumptions and constructions, nothing special to be stated.

\begin{assumption}
This is \texttt{\string\begin\stringall{assumption}\endstringall}.
\end{assumption}

\begin{assumption}
[name]
This is \texttt{\string\begin\stringall{assumption}[name]\endstringall}.
\end{assumption}

\begin{construction}
This is \texttt{\string\begin\stringall{construction}\endstringall}.
\end{construction}

\begin{construction}
[name]
This is \texttt{\string\begin\stringall{construction}[name]\endstringall}.
\end{construction}

\section
[\texorpdfstring
{Citations~\citenolink{ref_article1}}
{Citations [Aut16]}]
{Citations~\cite{ref_article1}}

For citations of references, Springer LNCS prefers the use of square brackets
and consecutive numbers. Citations using labels or the author/year
convention are also acceptable. The following bibliography provides
a sample reference list with entries for journal
articles~\cite{ref_article1}, an LNCS chapter~\cite{ref_lncs1}, a
book~\cite{ref_book1}, proceedings without editors~\cite{ref_proc1},
and a homepage~\cite{ref_url1}. Multiple citations are grouped
such as~\cite{ref_article1,ref_lncs1,ref_book1}
and~\cite{ref_article1,ref_book1,ref_proc1,ref_url1},
if the numeric style is used.

The \texttt{cite} package is loaded and \texttt{nomove,noadjust} are set.
Using \texttt{noadjust} removes the extraneous space
in case of \texttt{\string\begin\stringall{lemma}[\cite{ref}]\endstringall},
which is rendered as ``\texttt{Lemma 1 ([ABC])}''
instead of ``\texttt{Lemma 1 ( [ABC])}''.
Because of \texttt{noadjust},
no space is inserted for \texttt{\stringall word\cite{ref}\endstringall},
and it should be entered as
\texttt{\stringall word~\cite{ref}\endstringall}.
The \texttt{nomove} option does not manifest itself
for non-superscript citations.

Bugs of \texttt{\string\doi}
in some versions of \texttt{splncs04.bst}
are mitigated.

Use
\href{https://github.com/cryptobib/export}{\texttt{cryptobib}}
to make citing cryptography literature easier.
Use \texttt{\string\citenolink} to
cite something without making a hyperlink.

\chapter
[\texorpdfstring{% TeX
Chapter with Multiline Names
\tocchaplinebreak\tocchaphanging
and Hanging Indentation
\tocchaplinebreak
(See?)
}{% PDF string
Chapter with Multiline Names
and Hanging Indentation
(See?)}]
{Chapter with Multiline Names
\\and Hanging Indentation (See?)}

\long\def\myrepeat#1\relax{#1\relax#1\relax}
\long\def\myloop#1\relaxed{#1\relaxed#1\relaxed#1\relaxed}
\let\relaxed\relax
\newcounter{mychaptercounter}
\newcounter{mysectioncounter}

\myrepeat\myrepeat
\part{A new part! Hey!}
\myrepeat\myrepeat\myrepeat
\stepcounter{mychaptercounter}
\chapter{Additional Chapter \Roman{mychaptercounter}}
This is some content to make this section non-empty.
\setcounter{mysectioncounter}{0}
\myloop\myloop
\stepcounter{mysectioncounter}
\section{Additional Section \Roman{mychaptercounter}.\Alph{mysectioncounter}}
Who says no?!
\relaxed
\relax

\let\myrepeat\relax
\let\mychapter\relax
